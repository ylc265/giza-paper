\begin{figure*}
\begin{tabular}{c|c|c|c|c|c}
& coding rate & \# of DCs & DC location & Latency to Majority & Max Latency\\
\hline
US-2-1 & 2+1 & 4 & west, south central, central & ?? ms& ?? ms\\
US-6-1 & 6+1 & 8 & central, west central, north central, east1, east2, south central, west&?? ms&?? ms\\ 
world-2-1 & 2+1 & 4 & central, japan east, north europe &?? ms &?? ms\\
world-6-1 & 6+1 & 8 & central, west central, north central, japan east1, japan east2, uk west1, north europe &?? ms &?? ms\\
\end{tabular}
\caption{The DC configurations and inter-DC latencies in various experiments~\label{fig:dcconfig}} 
\end{figure*}

\subsection{Setup}
Our experiments involve a varying number of DCs in different configurations. In
each DC, we deploy a single Azure virtual machine (16 cores, 56 GB of RAM, and
gigabit ethernet) and create a storage account for accessing Azure blob and
table storage. Both the blob and table storage are configured with the
``locally redundant'' replication level.  Each \name node accesses the local
DC's cloud storage and also proxies the storage requests from \name nodes in
other DCs.  

For CockroachDB experiments, we run a CockroachDB cluster spanning across
multiple DCs.  We use the same set of Azure virtual machines and run a single
CockroachDB node per DC. Our configuration of CockroachDB follows the
recommended production settings by the developers of CockroachDB. For example,
we run NTP to synchronize the clocks of different CockroachDB nodes. 

We generate experimental workloads using the YCSB benchmark. In the generated
workload, the probability of accessing a given key follows a Zipf distribution.
We experiment with different object sizes ranging from 128KB to 4MB. 

We experiment with four diffferent DC configurations, as shown in
Figure~\ref{fig:dcconfig}.  These configurations correspond to different coding
rates and different choices of DCs, either within US-only or spread across the
world. The wide-area latency across different DCs plays an important role in 
the performance of \name.  Figure~\ref{fig:dcconfig} reports the majority
latency (measured as the latency required to get a response from a majority
quorum of DCs) and the maximum latency between DCs.

