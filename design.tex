\section{Design}

\subsection{Interface and Assumptions}
Giza is an erasure coding scheme for key/value stores and provide two operations: Put(Key, Value) and Get(Key), where Key and Value are arbitrary strings. Get(Key) returns the value of the latest Put for that key. (Add assumptions here + what giza is good for: Giza provides fault tolerance for non byzantine failures in an asynchronous network)

\subsection{Architecture Outline}
A typical giza architecture for a data center includes the giza nodes, the Azure Blob Storage, and the Azure Table Storage. The giza nodes are the processing units of the Giza architecture and manages the data and the metadata. Furthermore the giza nodes participate in paxos rounds as coordinators. Figure 1 illustrates the architecture of Giza. Giza separates data from metadata and handles them on different paths. The data path is responsible for encoding the data and sending the data fragments across data centers. Each data fragment is stored in the corresponding DC’s Azure Blob Storage. The metadata path is responsible for storing the latest version of the data and the location of its data fragments. Giza uses a variant of the Paxos state machine replication (SMR) to maintain consistency of the metadata where each metadata server maintains a local copy of the replicated log. The replicated log is stored in the Azure Table Storage.

\subsection{Metadata}
The metadata for each object includes the latest modification to that object using a versioning scheme. The highest version corresponds to the latest modification and the decided value for the version. In addition, the metadata also includes the location of the data fragments for the latest version. The metadata is replicated using a variant of the Paxos state machine replication. Instead of having a single log recording all the executions, each object has a corresponding paxos log. Giza uses the fault-tolerant Azure Table to store the metadata where each entry in the table corresponds to an object. Figure 2 illustrates the schema of the metadata table.

\subsection{Giza Workflow}
Because of the high WAN RTT, Giza’s primary goal is to minimize the number of round trips in its put and get critical paths while maintaining strong consistency semantics. By approaching this problem with multiple iterations, we were able to reduce the original 3 WAN RTT to 1 WAN RTT in most cases, dramatically reducing the put and get latency.

\subsubsection{3 RTT Put and Get}
Figure 3 illustrates the workflow of a typical Put operation. When a client issues a Put command, Giza first queries its local metadata to identify a most likely latest version of the object. It then starts a paxos round for the version. A paxos round is broken down into two phases: the prepare phase and the accept phase. Giza runs the data path first, where the object is erasure encoded and the data fragments are sent to the corresponding data centers. (Here, we have to write this information down in case of giza node failure right). When the data path returns successfully, Giza proceeds to run the metadata phase which incurs two round trips.
\par 
During the Get path, Giza runs the metadata phase first to get the location of the data fragments. To ensure a consistent latest version of the data, Giza runs a paxos round for get on the object’s entry log. When this succeeds, Giza runs the datapath and receives enough data fragments to reconstruct the original data before returning to the client. Both the Put and Get incurs two rounds in the metadata path and 1 round in the data path.

\subsubsection{2 RTT Put and Get}
We quickly realized that the separation of metadata path and data path allows for some parallelism. In particular, during the Put path, Giza can run the prepare phase of the metadata path in parallel with the data path. When prepare phase and the data path succeed, Giza runs the second phase of the paxos round. Once this round completes, Giza can return acknowledgement to the client. This brings the round trips down to 2 round trips in the optimal case.
\par
Furthermore, during Get, Giza can run the additional paxos round for the get operation in parallel with the data path get. This can be done by first optimistically assuming that the current highest entry of the object in its local entry log is in fact the highest version. Once this information is obtained, Giza can then proceed to get the data fragments while simultaneously validating the consistency of the latest version on the metadata path. If the assumption is correct, the Get incurs 2 round trips.

\subsubsection{1 RTT Put and Get}
We used the classical Paxos protocol for achieving consistency on our metadata path. [Introduce Fast Paxos and why this works in this case, should do this after reading in depth about Fast Paxos]. This reduces Giza’s put path to 1 round trip.
\par
In the Get Path, we realized that we can forego running the paxos round in some scenarios by including a learning phase on the non-critical path of Put. After acknowledging to the client’s put request, the coordinator giza node will send the accepted version and the version number as a committed entry for the object to the other participating giza nodes. This entry is used to serve as the highest committed version and will only override existing entry if the version is higher. During a client’s Get request, Giza first obtains the object’s row entry from the majority of the participating giza nodes. If the latest entry in the paxos log are consistent or not higher than the highest committed entry, then Giza can safely return the results obtained from the data path. However, in the case where the latest entry in the paxos log is higher than the highest committed entry (This can happen if the giza nodes are simultaneously performing a paxos round for a newer version of the object or the learning phase has not reached the giza nodes), then Giza can still get the highest version but doing two things. First it can fallback to the previous approach of running a paxos round for the get operation. Alternatively, it can wait to obtain the object’s row from all the participating giza nodes (instead of the previous majority). In our scheme, we execute both options concurrently and return once the faster of the two approaches completes.

\subsection{Giza Recovery}
Describe the mechanism of recovery when a data path node fails but a metadata path node doesn’t (6-1 scheme still only utilizes 3 giza nodes for metadata).
\par
Describe the mechanism when a DC containing both data path and metadata path node fails (membership change).


