\section{Evaluation}
Giza provides fault tolerance through erasure coding across wide area networks while providing linearizability for its read and write operations. As such, there are currently no system that is specifically designed to serve as an alternative to Giza. However, we benchmark Giza’s performance against Cassandra and CockroachDB to illustrate the following points:

- In the common case, Giza’s Fast Paxos implementation of the metadata phase allows for lower latency writes for Giza when compared to Cassandra.

- While one could implement Giza’s metadata phase and data phase as a single distributed transaction, we illustrate through CockroachDB that doing so will degrade performance significantly.

In addition, we provide evaluation results for encoding scheme ranging from 2-1 in 3 data centers to 18-4 in 22 data centers to illustrate the flexibility of Giza as a front end.
\subsection{Setup}
For all experiments, we deployed a single virtual machine (how many cores, how many whatever) for each geographical region to act as the front end for the regional data center. In addition, The virtual machines are deployed through the console provided by the cloud computing platform. 

How did we set up Giza?

We deployed both a table service and a blob service provided by the cloud platform for each geographical region. The granularity of replication for these services varies from provider to provider but we always choose the replication level to match that of the regional replication. This means that as long as there’s no dc outage, the data would not be lost. On top of each table service and blob service, we also run a front end interface for the table and blob service respectively. This is to avoid unnecessary WAN round trip when executing metadata path logic. The Giza node interacts with table and blob service front end via Thrift RPC calls, sending the appropriate metadata and data to other regions.  

How did we set up CockroachDB?

We run a CockroachDB cluster spanning all the data centers included in the experiment where each node correspond to the virtual machine running on each geographical region. We use a single database of CockroachDB to emulate our Giza read and write path. The database contains multiple tables, a metadata table and multiple data tables corresponding to their respective region, and the tables differ in their replication level. The metadata table is replicated according to the fault tolerance level, i.e 2 if the the cluster can tolerate one data center outage [what about the location?]. The data table is replicated once only at the node of the respective region.
*machines*

*workloads ycsb*

*performance metrics: latency, cpu*

*is it possible to choose real workload?*


\subsection{Small object}
64K $\sim$ 16MB

X-axis: Value size
Y-axis: 50\% Read latency

X-axis: Value size
Y-axis: 90\% Read latency

X-axis: Value size
Y-axis: 99\% Read latency

Same for write

[adding cpu results in a table]

\subsection{Large object}
256MB $\sim$ 1GB

X-axis: Value size
Y-axis: Average Read latency

X-axis: Value size
Y-axis: Average Write latency


\subsection{Contention}

Fixed object size
X-axis: zipf coefficient
Y-axis: 50\%, 90\%, 99\% Read/Write Latency


\subsection{Real workload}
Table.