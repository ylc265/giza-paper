\section{Evaluation}
Giza provides fault tolerance through erasure coding across wide area networks while providing linearizability for its read and write operations. As such, there are currently no system that is specifically designed to serve as an alternative to Giza. However, we benchmark Giza’s performance against Cassandra and CockroachDB to illustrate the following points:

- In the common case, Giza’s Fast Paxos implementation of the metadata phase allows for lower latency writes for Giza when compared to Cassandra.

- While one could implement Giza’s metadata phase and data phase as a single distributed transaction, we illustrate through CockroachDB that doing so will degrade performance significantly.

In addition, we provide evaluation results for encoding scheme ranging from 2-1 in 3 data centers to 18-4 in 22 data centers to illustrate the flexibility of Giza as a front end.
\subsection{Setup}
For all experiments, we deployed a single virtual machine (16 cores, 56 GB of RAM, 800 GB SSD, and gigabit ethernet) for each geographical region. We use the same virtual machines for setting up the Cassandra and CockroachDB clusters. The client issuing the requests runs on one of the virtual machine that is also part of the cluster. 
To set up Giza, we also had to deployed both a table service and a blob service provided by the cloud service platform. The granularity of replication for these services varies from provider to provider but we always choose the replication level to match that of the regional replication. This means that as long as there’s no dc outage, the data would not be lost. For each data center, we run a Giza node frontend with the virtual machine. The Giza node can service requests from a client running in the virtual machine. In addition, requests to its local table and blob storage from other Giza nodes also go through the Giza node frontend in the form of an RPC call. This is to avoid unecessary WAN round trips when dealing with complicated table and blob storage operations. 

How did we set up Cassandra?

This is how we set up Cassandra.

How did we set up CockroachDB?

We run a CockroachDB cluster spanning all the data centers included in the experiment where each node correspond to the virtual machine running on each geographical region. We use a single database of CockroachDB to emulate our Giza read and write path. The database contains multiple tables, a metadata table and multiple data tables corresponding to their respective region, and the tables differ in their replication level. The metadata table is replicated according to the fault tolerance level, i.e 2 if the the cluster can tolerate one data center outage [what about the location?]. The data table is replicated once only at the node of the respective region.
*machines*

We generate our workloads using the YCSB benchmark. Since our systems function as key value blob storage, we only require one field, the blob value, for each key. The probability that a given key will be accessed is given by the Zipfian distribution. We evaluated our systems with different blob value sizes. The smaller sizes include 64KB, 128KB, 256KB, and 512 KB while the bigger sizes include 1MB, 2MB, 4MB, and 8MB. We also evaluated them under different read/write ratios, including equal-shares and all writes.

\subsection{Read/Write Latency}

\subsection{Small object}
64K $\sim$ 16MB

X-axis: Value size
Y-axis: 50\% Read latency

X-axis: Value size
Y-axis: 90\% Read latency

X-axis: Value size
Y-axis: 99\% Read latency

Same for write

[adding cpu results in a table]

\subsection{Large object}
256MB $\sim$ 1GB

X-axis: Value size
Y-axis: Average Read latency

X-axis: Value size
Y-axis: Average Write latency


\subsection{Contention}

Fixed object size
X-axis: zipf coefficient
Y-axis: 50\%, 90\%, 99\% Read/Write Latency


\subsection{Real workload}
Table.