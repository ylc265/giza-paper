\subsection{Joint Optimization of Data and Meta-data Operations}

The data path of \name is straightforward: the \name node encodes the data to
$k$ original fragments and $m$ parity fragments. $k$ and $m$ are configurable.
Then the Giza node computes a content hash for each fragment, and uses the hash
value as the key to write each fragment to a separate data center. When naively
combining the data path with the earlier metadata path, the \name node
serializes the two paths, resulting in two or more cross-DC round trips. To
reduce latency, we optimize \name to run the data and metadata paths in
parallel. This is potentially problematic because either the data or metadata
path could fail while the other one succeeds. Below, we describe how {\em put}
and {\em get} cope with this challenge and ensure end-to-end correctness.

\comment{

{\name} supports three operations: put, get and delete. All of them require a
key as an argument to identify an object. put takes object content as an extra
argument, get returns object content (null if non-exist) as result. A delete
operation is processed as a special write, putting a tombstone in the object's
metadata. The actual recycling of disk space happens at garbage collection,
which will be described in ~\Cref{sec:xxxx}.

}

{\bf The \emph{put} Operation:} After generating the coded fragments and calculating
their content hashes, the \name node launches both the data and metadata paths
in parallel. In the common case, the \name node waits for both the data and the
metadata paths to finish before acknowledging clients. Furthermore, it
replicates the Commit confirmation only after both the data and the metadata
paths complete. In other words, \name ensures that {\tt known committed version}
only include those versions whose data and metadata have both been successfully
committed.

In one uncommon case, the data path succeeds, while the metadata path fails.
Now, the fragments stored in the cloud blobs become useless.
Giza deletes these fragments and reclaims storage through a cleaning process,
which first executes Paxos to update the current version to {\em no-op}
and then removes the fragments from the corresponding blob stores in all the DCs.

In another uncommon case, the data path fails, but the metadata path succeeds.
This is rather subtle, as it creates a challenging for the {\em get} operation,
as addressed next.

{\bf The \emph{get} Operation:}
A naive way to perform {\em get} is to first read the latest metadata and then retrieve the coded fragments.
To reduce latency, {\name} chooses an optimistic approach and parallelizes the metadata and the data paths.

For a {\em get} request, the {\name} node first reads from the local DC the corresponding metadata table row.
It obtains {\tt known committed version}, as well as the names and locations of the coded fragments of the latest version.
The \name node immediately starts reading the coded fragments from the different data centers.
Separately, it launches a regular metadata read to validate that the version is indeed the latest.
If the validation fails, the \name node realizes there is a newer version.
It in turn has to redo the data path by fetching a different set of coded fragments.
This results in wasted efforts in its previous data fetch.
Such potential waste, however, only happens when there is concurrent writes on the same object,
which is rare.

Because the data and metadata paths are performed in parallel during {\em put},
it is possible (though rare) that the coded fragments for the latest committed version
have not been written to the blob storage at the time of read.
This happens if the metadata path in the {\em put} finishes before the data path,
or the metadata path succeeds while the data path fails.
In such case, the {\name} node needs to fall back to read the previous version,
as specified in {\tt known committed version}.

\comment{

\sm {
  Hi Daniel, can you give me a few details about the hashing? e.g. hashing method,
  hashing result size, collision rate, etc.)
}

}

\subsection{Deletion and Garbage Collection}

The \emph{delete} operation in \name is treated as a special update of the
object's metadata. When receiving a delete request, \name will enter the
metadata path writing a new version as if responding to a normal put request,
except that it will put a flag in the metadata indicating the object is deleted.
Another difference is that it will not issue any writes to data fragments. Upon
future read request, when the system reads an object whose metadata has the
deletion flag set for the highest version, it will treat the object as deleted.

Other than handling the users' delete requests, \name also needs to support two
extra features regarding freeing storage space: 1) trimming earlier versions and
2) erasing an entire object thoroughly from the storage. To control the cost of
table and storage service, \name limits the number of versions for each object.
Once the threshold is reached, writing a new version will trigger the garbage
collection that trims the earliest version. After an object is deleted and no
new writes refresh the object, the space for its metadata and data also needs to
be recycled.

%Deleting a specific version is processed as a special update of the object's
%metadata. \name executes the Paxos algorithm to write a garbage collection
%record containing the version $v$; any version $v'\leq v$ will be garbage
%collected.

\name garbage collects and reclaims the storage space occupied by the coded
fragments of the deleted version. \comment{ Because {\name} keeps writing new
  version metadata to table service and data fragments to blob storage service,
  it needs to garbage collect on outdated versions to recycle storage space.
  Moreover, the garbage collection also needs to free the space marked as
  tombstone by the delete operation. } The garbage collection process includes
deleting the coded fragments and truncating the column for the deleted version
in the metadata table row. It follows three steps: 1) fetch the metadata
corresponding to the version to be garbage collected. 2) delete the coded
fragments in the blob storage. 3) remove the column for the deleted version from
the metadata table row. The second step has to occur before the third one in
case that the garbage collection process is interrupted and the coded fragments
may become ``orphans'' without proper metadata pointing to them in the table
storage.

Reclaming all space of an deleted object requires extra care, due to possible
contention from the \emph{put} requests. If removed brutally, a new {\em put}
operation of the same object may lead the system into an abnormal state. For
instance, the {\em put} operation could start at a data center where the table
row is already deleted. The {\em put} operation will therefore assume the object
never existed before and then choose the smallest version number. Committing
this version number is dangerous before the metadata table row is deleted from
all the DCs, as this may result in conflicts and ambiguity during future failure
recovery.

%To delete an entire object, {\name} first executes Paxos to commit a tombstone
%as the highest version. Then, it deletes the coded fragments corresponding to
%all the versions of the object. Finally, it starts to delete the entire table
%row from the tables across all the DCs.


% start with an initial version, which could violate some of the metadata (with
% a higher version before the deletion) if the removing is still in process.
Therefore, {\name} resorts to a two-phase commit protocol to reclaim the space
for the metadata table row after trimming all its previous versions. In the
first phase, it marks the rows in all the data centers as \texttt{confined}.
After this any other {\em get} or {\em put} operations are temporarily disabled
for this object. Then in the second phase, all the rows are actually removed
from the table storage. The disadvantage of this approach is obvious. It
requires all the data centers to be online. Data center failure or network
partition may pause the process and make the row unavailable (but can still
continue after data center recovers or network partition heals).  



% \subsection{Giza Workflow}
% Because of the high WAN RTT, Giza’s primary goal is to minimize the number of round trips in its put and get critical paths while maintaining strong consistency semantics. By approaching this problem with multiple iterations, we were able to reduce the original 3  WAN RTT to 1 WAN RTT in most cases, dramatically reducing the put and get latency.

% \subsubsection{3 RTT Put and Get}
% Figure 3 illustrates the workflow of a typical Put operation. When a client issues a Put command, Giza first queries its local metadata to identify a most likely latest version of the object. It then starts a paxos round for the version. A paxos round is broken down into two phases: the prepare phase and the accept phase. Giza runs the data path first, where the object is erasure encoded and the data fragments are sent to the corresponding data centers. (Here, we have to write this information down in case of giza node failure right). When the data path returns successfully, Giza proceeds to run the metadata phase which incurs two round trips.
% \par 
% During the Get path, Giza runs the metadata phase first to get the location of the data fragments. To ensure a consistent latest version of the data, Giza runs a paxos round for get on the object’s entry log. When this succeeds, Giza runs the datapath and receives enough data fragments to reconstruct the original data before returning to the client. Both the Put and Get incurs two rounds in the metadata path and 1 round in the data path.

% \subsubsection{2 RTT Put and Get}
% We quickly realized that the separation of metadata path and data path allows for some parallelism. In particular, during the Put path, Giza can run the prepare phase of the metadata path in parallel with the data path. When prepare phase and the data path succeed, Giza runs the second phase of the paxos round. Once this round completes, Giza can return acknowledgement to the client. This brings the round trips down to 2 round trips in the optimal case.
% \par
% Furthermore, during Get, Giza can run the additional paxos round for the get operation in parallel with the data path get. This can be done by first optimistically assuming that the current highest entry of the object in its local entry log is in fact the highest version. Once this information is obtained, Giza can then proceed to get the data fragments while simultaneously validating the consistency of the latest version on the metadata path. If the assumption is correct, the Get incurs 2 round trips.

% \subsubsection{1 RTT Put and Get}
% We used the classical Paxos protocol for achieving consistency on our metadata path. [Introduce Fast Paxos and why this works in this case, should do this after reading in depth about Fast Paxos]. This reduces Giza’s put path to 1 round trip.
% \par
% In the Get Path, we realized that we can forego running the paxos round in some scenarios by including a learning phase on the non-critical path of Put. After acknowledging to the client’s put request, the coordinator giza node will send the accepted version and the version number as a committed entry for the object to the other participating giza nodes. This entry is used to serve as the highest committed version and will only override existing entry if the version is higher. During a client’s Get request, Giza first obtains the object’s row entry from the majority of the participating giza nodes. If the latest entry in the paxos log are consistent or not higher than the highest committed entry, then Giza can safely return the results obtained from the data path. However, in the case where the latest entry in the paxos log is higher than the highest committed entry (This can happen if the giza nodes are simultaneously performing a paxos round for a newer version of the object or the learning phase has not reached the giza nodes), then Giza can still get the highest version but doing two things. First it can fallback to the previous approach of running a paxos round for the get operation. Alternatively, it can wait to obtain the object’s row from all the participating giza nodes (instead of the previous majority). In our scheme, we execute both options concurrently and return once the faster of the two approaches completes.

% \subsection{Giza Recovery}
% Describe the mechanism of recovery when a data path node fails but a metadata path node doesn’t (6-1 scheme still only utilizes 3 giza nodes for metadata).
% \par
% Describe the mechanism when a DC containing both data path and metadata path node fails (membership change).

%%% Local Variables:
%%% mode: latex
%%% TeX-master: "main"
%%% End:

