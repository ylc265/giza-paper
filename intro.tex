\section{Intro}
Giza is a cloud storage front end that erasure codes large mutable data files across multiple data centers . The motivation of Giza is to reduce overall storage cost and to tolerate entire data center failures, while maintaining strong consistency and reasonable read and write latencies.
\par
We have made the following contributions in this paper. First, we observed that the growth in network bandwidth will soon surpass the growth in storage capacity and that cross WAN erasure coding can and should be realized in practice. Second, we have demonstrated feasibility of such a system by building a cross WAN erasure coding key value store using limited interface supported by many existing cloud APIs. This key value store is strongly consistent and can tolerate complex network partitions. Lastly, we provide some initial latency results of our system running on top of an existing cloud platform to show that such a system can replace current solutions for certain workloads.

\subsection{Background}
Paxos is a consensus algorithm that satisfies the safety property in an asynchronous network. 
\par
Erasure Encoding is a very mature technique used in storage systems for data striping and fault tolerance.
