\section{Related Work}

%%% Local Variables:
%%% mode: latex
%%% TeX-master: "main"
%%% End:

{\bf Erasure coded storage.}
There are two general approaches in applying erasure coding to storage.  One
is to generate coded fragments within each data object.  This is commonly used
to achieve redundancy within a single data center~\cite{facebook:f4}.  Another 
approach is to treat multiple data objects as a coding group and generate
parity blocks that combine multiple objects.  Facebook's cross-DC erasure
coding uses this scheme to code immutable blobs.  To handle mutable coded
blocks, prior work resort to a RAID-like approach~\cite{haibo:fast} of using
static coding groups comprising of fixed sized blocks. The RAID approach has
only been applied in cluster settings.


{\bf Consistency in mutable storage.}

