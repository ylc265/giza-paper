\section{Failure Recovery}

There are two categories of failures: node failures and DC failures.
The cloud table service and blob storage service that {\name} depends upon have already 
been designed and tested to be resilient in the face of machine failures within a DC. 
Those services have also been deployed and extensively tested. 
Since each \name node is stateless, handling node failure is simple: we can
simply launch new machines to replace any failed ones. In this section we focus
on how \name handles DC failures.

{\bf Transient DC failure.}
We categorize the transient DC failure to include temporary outages of the blob
and table storage service in a DC. 
When an entire DC becomes unavailable, such as caused by a
temporary network partition or power failure, by design, {\name} can still
serve Get/Put requests, albeit at degraded performance. 
For example, when handling a Put,
\name takes two cross-DC roundtrips instead of one, because there are not enough
DCs to make up a fast path quorum when some of DCs replicating the metadata
are down. 
% When handling a Get, \name must read the parity fragment as well as
% data fragments to re-construct the entire data object.
% JinL: Get also needs to make sure metadata is the correct version, ref to the last section. 

When a datacenter comes back online after crash, it needs to update the metadata
rows in its table service and data fragments in the blob storage. A number of
recovering clients are launched to scan through the objects. For each object, it
issues a normal read request except that it does not pre-fetch the data. If the
local version match the global version, nothing needs to be done; if the local
version is behind, the recovering process then reads the latest data fragments,
re-calcuates the fragment that belongs to the local DC and writes it to
the blob storage consistently with the metadata.

\comment{

\sm {
JinL: What about new objects created during the transient failure of DCs. I felt that there need to be log that tracks object creation, as those objects can not be discovered by just scanning the local metadata table. Scanning remote metadata table is another option (expensive though). 
}
}

{\bf Permanent DC Failure}
Although extremely rare, a data center may fail catastrophically. The blob and
table storage service within the DC may also experience long-term outages. We
categorize these all as permanent DC failure. 

%\name handles permanent DC failure by replacing the failed
%DC with a healthy new one. Many failure recovery coordinators are launched to
%reconstruct the data fragments and re-insert the metadata row. This procedure is
%similar to how it handles transient failures yet may last much longer (weeks)
%because it needs to recover all objects. This is acceptable considering
%permanent data center failure happens rarely, perhaps only once in decades.
%
%To ensure all \name nodes will read and write through the new DC instead of the
%old ones, we have a reliable DNS service storing the mapping from data center
%ids to actual ip addresses. Each \name node maintains a {\em lease} with the DNS
%service. The lease usually lasts for a time period, e.g. 30 minutes. Before the
%lease expires the node will renew the lease by refreshing its local DNS cache.
%During permanent DC failure recovery, we first update the mapping in the DNS
%service, then wait for a long enough time (e.g. $>$30 minutes) to make sure all
%giza nodes have the new mappings. The replicated DNS service already exists in
%many proprietary systems (e.g. Chubby~\cite{chubby:osdi06}) and has many open-source
%alternatives (e.g. ZooKeeper~\cite{zookeeper:atc10}) thus does not bring extra 
%implementation overhead to \name.

\name handles permanent DC failure by employing logical DC names in storage account. The mapping between a logical DC name to a physical DC location is stored in a separate service external to \name. Upon a permanent DC failure, the same logical DC name is re-mapped from the failed DC to a healthy new one. Giza metadata records logical DC names and therefore remains unchanged after the re-mapping. This is similar to DNS, where same domain name can be re-mapped to a different physical IP address. This way of handling failure is also reported in Chubby~\cite{chubby:osdi06}.

Upon the permanent DC failure, \name launches recovery coordinators to reconstruct the missing fragments and re-insert the metadata rows in the replacement DC. The procedure is similar to how \name handles transient failures yet may last much longer. This is acceptable considering permanent DC failure is uncommon after all.

%% \section{Replica Configuration Failure Recovery}

%% This section discusses two practical aspects of a {\name} deployment, namely,
%% how to configure which set of data centers to use for storing coded data
%% fragments and replicating metadata, and how to handle failures.

%% \subsection{Data center configuration}
%% \label{subsec:config}

%% \name supports a large number of paying customers, each of which is associated
%% with his/her own storage account.  Each \name account is configured with a
%% collection of three or more DCs that replicate the metadata and/or store the
%% coded data framents.  The customer specifies a desired fault tolerance goal, and an acceptable cost
%% of storing the data. The configuration determines the number of DCs that the customer's data will be stored, which in turn determines coding rate. For example, if the customer wishes to protect against
%% one data center failure and is willing to pay $17\%$ extra for the protection,
%% a 6+1 erasure code is to be used, and there should be 7 DCs in the configuration.  
%% Furthermore, 3 among the 7 DCs are designated as metadata DCs that replicate 
%% metadata in addition to storing coded data.
%% The choice of DCs is determined by user preferences. For
%% example, OneDrive could choose the DC where the user makes the most data access and 
%% other closeby DCs for better peformance.

%% The mapping from each \name account to its set of DCs is stored in a
%% separate configuration service external to the \name system.  To service
%% Get/Put requests for an account, \name retrieves the account's DC configuration
%% from the external service and caches the information for future requests for a
%% threshold period of time.

%% {\bf Changing DC Configuration.} 
%% The set of DCs for an account may be changed, either due to changing user
%% preferences or recovering from a failed DC.  Each new DC configuration for an
%% account is associated with a monotonically increasing identifier, view-id.
%% Since configuration changes infrequently, we keep the history of all DC configurations.  The view-id is
%% attached to each version of the object metadata. To change to a new DC
%% configuration for an account, \name enters a grace period during which the
%% cached DC configuration information in all {\name} nodes are invalidated.
%% During the grace period, it is possible that some nodes use the old DC configuration
%% while others use the new one. To ensure metadata consistency in this scenario,
%% whenever \name writes a new version of the metadata whose previous version has
%% the old DC configuration, its Paxos implementation obtains quorums in {\em
%% both} the old and new configuration.  Therefore, DCs in the old and new
%% configuration must concurrently agree on the same metadata version.

%% The migration of coded data fragments from the old configuration of DCs to the
%% new ones can happen outside of the grace period and in the background. Whenever
%% an object's fragments have been moved or re-generated in the new DCs, the
%% migration worker writes a new version of metadata for the object.
%% Before the completion of data migration, it is possible that a {\em Get} operation 
%% fails to obtain metadata or data in the new configuration.  In this situation, 
%% {\name} retries the operation under the previous configuration.

%% \subsection{Failure Handling}

%% There are two categories of failures: {\name} node failures and DC failures.
%% Since each \name node is stateless, handling node failure is simple: we can
%% simply launch new machines to replace any failed ones.  We focus on how \name
%% handles DC failures.

%% {\bf Transient DC failure.}
%% We categorize the transient DC failure to include temporary outages of the blob
%% and table storage service in a DC.  Giza copes with transient DC failure by
%% design.


%% The table service and blob storage service are reliable in the face of machine
%% failures within a DC.  When an entire DC becomes unavailable, such as a
%% temporary network partition or an entire datacenter breakdown, {\name} can
%% still serve Get/Put requests, albeit at degraded performance. When handling a Put, 
%% \name takes two cross-DC roundtrips instead of one,
%% because there are not enough DCs to make up a FastPaxos quorum when 1 out of 3 DCs
%% replicating the metadata is down.  When handling a Get, \name must read the parity 
%% fragment as well as data fragments to re-construct the entire data object.


%% When a datacenter comes back online after crash, it needs to update the
%% metadata rows in its table service and data fragments in the blob storage.
%% A number of recovering clients are launched to scan through its objects.
%% For each object, it issues a normal read request except that it does not
%% pre-fetch the data. If the local version match the global version, nothing
%% needs to be done; if the local version is behind, the recovering process
%% then reads the latest data fragments, re-calcuates the fragment that belongs
%% to the local datacenter and writes it to the blob storage consistently with
%% the metadata.


%% {\bf Permanent DC Failure}
%% Data center may fail catastrophically. The blob and table storage service
%% within the DC may also experience long-term outages. We categorize these all as
%% permanent DC failure.

%% \name handles permanent DC failure by triggering a DC configuration change 
%% which replaces the failed DC with a new healthy one. This process is identical 
%% to that required for the configuration change (Sec~\ref{subsec:config}) except
%% \name also need to re-generate lost coded fragments instead of simply moving
%% existing ones.  Furthermore, we prioritize work done to handle permanent DC
%% failure over that for ordinary DC configuration change.


%%% Local Variables:
%%% mode: latex
%%% TeX-master: "main"
%%% End:
